\documentclass[a4paper,twoside,titlepage,openright]{book}
\usepackage[MeX]{polski}
\usepackage[utf8]{inputenc}
\usepackage{enumitem} % słownik pojęć
\usepackage{amsmath}
\usepackage{tabularx} % tabele
\usepackage[usenames,dvipsnames,svgnames,table]{xcolor} % kolory jak~się chce gdzieś użyć
\usepackage{graphicx} % żeby ryciny i~zdjęcia były
\usepackage{listings} % syntax highlighting
\usepackage{verbatimbox} % marginesy dla~tabel
\usepackage{emptypage} % usuwa nagłówki i~numery stron z~pustych stron
\usepackage{afterpage} % to zapobiega ustawianiu obrazka PO tym

% PAGE LAYOUT
%\usepackage{showframe} % debug
\marginparwidth 0pt
\marginparsep 0pt
\usepackage[top=3.5cm,bottom=3.5cm,inner=3.5cm,outer=2.5cm]{geometry}

% HEADER, FOOTER
\usepackage{fancyhdr} 
\pagestyle{fancy}

% TABLE OF CONTENTS

%kropki w~spisie tresci
\makeatletter
\def\numberline#1{\hb@xt@\@tempdima{#1.\hfil}}
\makeatother

% CHAPTER TITLE

%kropki po~tytułach rodziałów
\makeatletter
\def\@makechapterhead#1{%
  \vspace*{50\p@}%
  {\parindent \z@ \raggedright \normalfont
	\ifnum \c@secnumdepth >\m@ne
	  \if@mainmatter
	   \huge\bfseries \@chapapp\space \thechapter.
	   \par\nobreak
	   \vskip 20\p@
	\fi
   \fi
   \interlinepenalty\@M
   \Huge \bfseries #1\par\nobreak
   \vskip 40\p@
  }}
\makeatother

% SPIS TREŚCI

%kropki w~spisie tresci
\makeatletter
\def\numberline#1{\hb@xt@\@tempdima{#1.\hfil}}
\makeatother

% TYTUŁY ROZDZIAŁÓW

%kropki po~tytułach rozdziałów
\makeatletter
\renewcommand*\@seccntformat[1]%
{\csname the#1\endcsname.\enspace}
\makeatother


% KONFIGURACJA WYGLĄDU NAGŁÓWKA TEGO CO SIĘ POWTARZA

\fancyhead{} 
\fancyhead[LE]{\rightmark}
\fancyhead[RO]{\leftmark}

% WYGLĄD TABEL

% vertical padding
\renewcommand{\arraystretch}{1.5}

% CODE LISTINGS 

\definecolor{mygreen}{rgb}{0,0.6,0}
\definecolor{mygray}{rgb}{0.5,0.5,0.5}
\definecolor{mymauve}{rgb}{0.58,0,0.82}

\lstset{ %
%frame=lines,
aboveskip=1.5em,
    belowcaptionskip=1.5em,
    xleftmargin=0.5cm,
  backgroundcolor=\color{white},   % choose the background color
  %basicstyle=\footnotesize,        % size of fonts used for the code
  breaklines=true,                 % automatic line breaking only at whitespace
  captionpos=b,                    % sets the caption-position to bottom
  commentstyle=\color{mygreen},    % comment style
  escapeinside={\%*}{*)},          % if you want to add LaTeX within your code
  keywordstyle=\color{blue},       % keyword style
  stringstyle=\color{mymauve},     % string literal style
}

\definecolor{maroon}{rgb}{0.5,0,0}
\definecolor{darkgreen}{rgb}{0,0.5,0}

\lstdefinelanguage{XML}
{
  basicstyle=\ttfamily,
  morestring=[s]{"}{"},
  morecomment=[s]{?}{?},
  morecomment=[s]{!--}{--},
  commentstyle=\color{darkgreen},
  moredelim=[s][\color{black}]{>}{<},
  moredelim=[s][\color{red}]{\ }{=},
  stringstyle=\color{blue},
  identifierstyle=\color{maroon},
  morekeywords={Page.DataContext,viewModel:NameViewModel}
}

%\setmonofont{Consolas} %to be used with XeLaTeX or LuaLaTeX
\definecolor{bluekeywords}{rgb}{0,0,1}
\definecolor{greencomments}{rgb}{0,0.5,0}
\definecolor{redstrings}{rgb}{0.64,0.08,0.08}
\definecolor{xmlcomments}{rgb}{0.5,0.5,0.5}
\definecolor{types}{rgb}{0.17,0.57,0.68}

\lstset{language=[Sharp]C,
%captionpos=b,
%numbers=left, %Nummerierung
%numberstyle=\tiny, % kleine Zeilennummern
%frame=lines, % Oberhalb und unterhalb des Listings ist eine Linie
showspaces=false,
showtabs=false,
breaklines=true,
showstringspaces=false,
breakatwhitespace=true,
escapeinside={(*@}{@*)},
commentstyle=\color{greencomments},
morekeywords={partial, var, value, get, set},
keywordstyle=\color{bluekeywords},
stringstyle=\color{redstrings},
basicstyle=\ttfamily\small,
}




\begin{document}

% ################################
%        STRONA TYTUŁOWA
% ################################

\begin{titlepage}

%\newgeometry{inner=3cm,outer=3cm}

\vspace*{1cm}
\begin{center}
\begin{Large}
Uniwersytet Mikołaja Kopernika\\[1mm]
Wydział Matematyki i~Informatyki\\[1mm]
\end{Large}
\end{center}

\vfill

\begin{center}
{\Large Klaudia Augustyńska}\\
nr albumu: 265408\\
informatyka
\end{center}

\vfill

\begin{center}
{\Large Praca magisterska}
\end{center}

\vspace{0.5cm}

\begin{center}
{\Huge \textbf{Wykorzystanie Cloud Computing w aplikacjach mobilnych}}
\end{center}

\vspace{2cm}
\hfill
\begin{minipage}{6.5cm}
Opiekun pracy dyplomowej\\
dr Błażej Zyglarski
\end{minipage}

\vfill

\begin{center}
Toruń 2018
\end{center}

\end{titlepage}

% odwracamy kartkę ze~stroną tytułową to nic nie~ma z~drugiej strony -> pusta strona
\clearpage{\pagestyle{empty}\cleardoublepage}

\tableofcontents
 
\chapter*{Wstęp}
\markboth{}{Wstęp}
\addcontentsline{toc}{chapter}{Wstęp}

\section*{identyfikacja problemu i stan wiedzy na ten temat}
\addcontentsline{toc}{section}{identyfikacja problemu i stan wiedzy na ten temat}

\section*{cel pracy}
\addcontentsline{toc}{section}{cel pracy}

\section*{opis rozdziałów}
\addcontentsline{toc}{section}{opis rozdziałów}


\clearpage{\pagestyle{empty}\cleardoublepage}
\chapter{Wprowadzenie}

\section{Krótkie naprowadzenie gdzie jesteśmy, w pigułce}



\section{Naprowadzenie jak doszło do tego że jesteśmy gdzie jesteśmy.}



\section{Doprecyzowanie czego konkretnie oczekujemy od współczesnej chmury.}




\section{Dokąd zmierza rozwój CC.}


\chapter{Porównanie różnych podejść}



\section{Chmury publiczne}

\section{Chmury prywatne}


\chapter{Analiza wyboru platformy dla aplikacji mobilnej}

\section{Wytumaczenie czemu to jest typowy projekt}

\section{Wymagania co trzeba wziąć pod uwagę}

\section{Werdykt, kandydatka nr 1, 2 ,3}



\chapter{Opis wdrożenia aplikacji zgodnie z kandydatką nr 1 }
 
\chapter*{Podsumowanie}
 
 
 
 
\addcontentsline{toc}{chapter}{Spis rysunków}
\listoffigures


\addcontentsline{toc}{chapter}{Bibliografia}
\begin{thebibliography}{99}

%\bibitem{inwestycje} \textsc{Jajuga K., Jajuga T.:}
%\textit{Inwestycje. Instrumenty finansowe, ryzyko finansowe, inżynieria finansowa}, Warszawa, Wydawnictwo Naukowe PWN SA, 2015, Wydanie III zm., \\ISBN 978-83-01-14957-4.

%\bibitem{soros} \textit{Billionaire who broke the Bank of England}, 
%\texttt{http://www.telegraph.co.uk/finance/2773265/\\Billionaire-who-broke-the-Bank-of-England.html}, dostęp: 10.12.2016.


\end{thebibliography}




\end{document}
